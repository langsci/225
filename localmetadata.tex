\author{Michael Daniel\and Nina Dobrushina\lastand Dmitry Ganenkov} %use this field for editor. 
\title{The Mehweb language}  
\subtitle{Essays on phonology, morphology and syntax}
\BackTitle{The Mehweb language} % Change if BackTitle is different from Title

\renewcommand{\lsSeries}{loc} % use lowercase acronym, e.g. sidl, eotms, tgdi
\renewcommand{\lsSeriesNumber}{1} %will be assigned when the book enters the proofreading stage

\BackBody{This is an investigation into the grammar of Mehweb (Dargwa, East Caucasian aka Nakh-Daghestanian), a lect of one village community in Daghestan, Russia, with a population of some 800 people, based on several years of team fieldwork. In many ways, Mehweb is a typical East Caucasian language, including a rich inventory of consonants, an extensive system of spatial forms in nouns and converbs and volitional forms in verbs, pervasive gender-number agreement, ergative alignment in case marking and in gender agreement. It is also a typical language of the Dargwa branch, with symmetrical verb inflection in the imperfective and perfective paradigm and extensive use of spatial encoding for experiencers. Although Mehweb is clearly close to the northern varieties of Dargwa, it has been long isolated from the main body of Dargwa varieties by speakers of Avar and Lak. As a result of both independent internal evolution and contact with its neighbours, Mehweb developed some deviant properties, including accusatively aligned egophoric agreement, a split in the feminine class, and the typologically rare grammatical categories of verificative and apprehensive. But most importantly, Mehweb is where our friends live.
% This is a first book-size English investigation in the grammar of Mehweb, a lect of one-village community in central Daghestan with a population of some 800 people. Mehweb belongs to the Dargwa branch of the East Caucasian language family. Its linguistic status as a dialect or language is debatable. Although Mehweb is clearly close to the northern varieties of Dargwa, it has been isolated from the main body of Dargwa varieties for at least several hundred years and underwent strong influence from its linguistic neighbours from other branches of the family, Avar and Lak. This resulted in some peculiar properties, both as a result of language internal development and language contact. Thus, Mehweb features egophoric agreement (which has clearly originated as reinterpretation of personal agreement in other Dargwa languages), a split in the feminine class (probably as a result of structural alignment with its neighbour, Lak), and the categories of verificative (a typological rarissima present in several other languages of the family but so far unattested in any other language of the world) and apprehensive (so far poorly attested in other East Caucasian languages). Other, more common features, are for the first time treated at this level of detail. This includes segmental effects of pharyngealization, the system of volitional moods, the classification of predicates according to their agreement patterns and the use of the assertive particle. The book is based on several years of collective fieldwork in the village carried out by a group of researchers and BA students of School of Linguistics, HSE (Moscow).
}

%\dedication{Change dedication in localmetadata.tex}
\typesetter{Vadim Radionov}
\proofreader{Ahmet Bilal Özdemir,
Andreas Hölzl,
Bev Erasmus,
Eva Schultze-Berndt,
Havenol Schrenk,
Ivica Jeđud,
Jeroen van de Weijer,
Jean Nitzke,
Sebastian Nordhoff,
Steven Kaye,
Sune Ryg{\aa}rd
Tatiana Philippova,
Yury Lander
}

\renewcommand{\lsID}{225} % contact the coordinator for the right number
\BookDOI{10.5281/zenodo.3374730}%ask coordinator for DOI
\renewcommand{\lsISBNdigital}{978-3-96110-205-1}
\renewcommand{\lsISBNhardcover}{978-3-96110-206-8} 
